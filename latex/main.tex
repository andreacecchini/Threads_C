\documentclass{article}

% Import packages 
\usepackage[utf8]{inputenc}

% Title 
\title{Thread e programmazione Multi Threading in C}
\author{Andrea Cecchini}

\begin{document}
    % Let visualize the title on the first page
    \maketitle
    % Introduction
    \section{Thread}
    % DEFINITION: Thread 
    Un \textbf{Thread} e' un componente di un processo che astrae
    il concetto di \textbf{flusso di istruzioni} che lo scheduler puo'
    far eseguire separatamente o \textbf{concorrentemente} con il resto
    del processo.
    \\ 
    Piu' banalmente possiamo pensare ad un Thread come ad una procedura
    che lavora in parallelo con le altre procedure del processo.
    
    % Strutture dati di un thread e condivisione della memoria
    \subsection{Contesto di un thread e condivisione della memoria}
    Un Thread per poter svolgere il suo lavoro si munisce di proprie 
    strutture dati.
    \\
    In particolare un thread ha un proprio \textbf{contesto}:
    \begin{itemize}
        \item Process ID
        \item Program Counter
        \item Stato dei registri
        \item \textbf{Stack di memoria}
        \item \dots
    \end{itemize} 
    Tuttavia i thread \textbf{condividono} alcune parti del proprio
    contesto, come:
    \begin{itemize}
        \item Zona Codice della memoria dedicata al processo
        \item Zona Data della memoria dedicata al processo nel quale e' 
        possibile accedere a variabili globali in maniera condivisa.
        \item Tabella dei File Descriptors
    \end{itemize}
    In questa maniera, oltre a condividere la CPU, ogni thread puo'
    accedere a tutte le variabili globali di un processo e anche alla
    tabella dei file descriptors.

    % Gestione dello scheduler di un thread 
    \subsection{Cambio di contesto e Scheduler}
    Ruolo centrale nella gestione dei thread viene svolta dallo
    \textbf{Scheduler}.
    Quello che fa non e' tanto diverso dalla suddivisione del tempo di 
    utilizzo della CPU con i processi, andiamo quindi ad estendere il
    concetto.
    Lo Scheduler affida per un certo \( \Delta t \) di tempo la CPU
    ad un \textbf{thread di un processo}.
    Successivamente lo scheduler potra' decidere di affidare la CPU
    ad un thread di un altro proceso o \textbf{dello stesso processo}.
    \\
    Questo fatto porta a delle considerazioni da fare.
    \\
    Il cambio di contesto tra threads dello stesso processo \textbf{e'
    molto piu' veloce} del cambio di contesto da parte di thread 
    appartenti a processi diversi.
    \\
    Questo fatto e' dovuto al fatto che i thread appartenti ad uno stesso
    processo condivino parte del contesto, quindi alcune parti rimaranno
    inalterate nella fase di switching.
    % Vantaggi/svantaggi dell'utilizzo dei thread al posto dei processi
    \subsection{Vantaggi/Svantaggio Thread vs Processi}
    Esploriamo i \textbf{vantaggi} nell'utilizzo di piu' thread
    anziche utilizzare piu' processi:
    \begin{itemize}
       \item Visibilita' dei dati globali
       \item Piu' flussi di esecuzione  
       \item Comunicazioni veloci, infatti ogni thread condivide lo 
        stesso spazio di indirizzamento, quindi le comuinicazioni fra i
        thread sono piu' veloci rispetto alle comuinicazioni fra i processi.
       \item Facilita' nella gestione degli eventi asincroni
       \item Context Swithing veloce 
    \end{itemize}
    Il fatto e' che alcuni vantaggi possono essere visti come dei \textbf{svantaggi}:
    \begin{itemize}
       \item Non si parla di parallelismo ma di \textbf{concorrenza},
        in quanto bisogna gestire il problema delle \textbf{mutua 
        esclusione} dei thread verso ai dati condivisi.
    \end{itemize}
    \textbf{ATTENZIONE!}, non si parla solamente di mutua esclusione dei
    dati globali di un processo, la concorrenza deve essere gestita da:
    \begin{itemize}
       \item Dai programmatori che scrivono il programma 
       \item Dal \textbf{Sistema Operativo} che deve implementare le 
        funzioni di libreria e le system calls in maniera \textbf{thread safe}.
    \end{itemize}
    Per introdurre il concetto di \textbf{rientranza e thread safe call},
    bisogna definire il significato di \textbf{operazione atomica}.
    % Atomicità delle operazioni
    \subsection{Atomicita' delle operazioni}
        % DEFINIZIONE: Operazione atomica
        In generale, un'operazione si dice \textbf{atomica} se risulta
        essere \textbf{indivisibile}, ovvero che un'altra operazione
        che utilizza dei dati condivisi con essa non puo' iniziare prima
        che sia finita la prima.
        In altre parole non si puo' utilizzare la caratteristica dell'
        \textbf{interleaving} che mette a disposizione la CPU.
        Una conseguenza di questo fatto e' che \textbf{a parita' di condizioni iniali, un'operazione atomica portera' al medesimo
        output}.
        Parlando pero' di operazioni che accedono alla memoria e che 
        potrebbero invocare lo \textbf{swapping} e il \textbf{page fault}
        dobbiamo estendere il significato:
        \\
        \textbf{Un'operazione si dice atomica se a parita' di input otteniamo lo stesso output}.        
        \\
        Questo comporta che un'operazione atomica puo' essere interotta per eseguirne un'altra piu' importante, per poi essere \textbf{richiamata e startata dall'inizio}, magari con un cambio di input iniziale.
\end{document}
